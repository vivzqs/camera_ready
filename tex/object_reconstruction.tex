\mysubsubsection{Object reconstruction}
The rule takes a room, whose shape has been reconstructed, that is, a
room node with out-going directed edges (pre-condition), then segments
the remaining unexplained 3D points, and generates object nodes
(transformation). Note that a point-cloud is used as the geometric
representation, as the scene visualization is the key application and a
point-based rendering is suitable for incomplete objects rather than a
mesh.
%Object-scale scene understanding and reconstruction has been actively
%studied in the past, where one can simply replace the reconstruction
%algorithm in our framework.

\yasu{given also details}
Given a room node and its extruded 3D model, we first collect 3D points
that are inside the room but not too close to the existing
geometry with a margin of 0.1 times the room height.
% More concretely, we discretize the 3D room space by a voxel grid, so
% that the maximum dimension spans $64$ voxels.  A voxel is marked as $1$,
% if it intersects with the room geometry. After dilating the values in
%the voxel grid twice,
%by a few iterations (2 in our experiments),
A variant of a standard point cloud segmentation algorithm
PCL~\cite{PCL} is used to segment the point-clouds. For better rendering
effects, we remove small objects that contain less than 1000 points, as
well as floating objects whose minimum Z-coordinate is above the room
center.

%where the minor modifications are provided in the supplementary
%material.

% %Further refinements for segmented objects are necessary to get good
% %rendering effect. First,
% we remove small noisy objects which contain
% less than 1000 points, as well as floating objects whose minimum z
% coordinate is greater than half of the room height.

% \mysubsubsection{Room merging} The rule takes two reconstructed room
% nodes if the room connection classifier (explained in detail below)
% identifies the connection type to be ``open'' (pre-condition),  the
% grammar simply removes the two rom nodes, and replace it with a single
% room node (transformation).

% \mysubsubsection{Door addition} The rule takes two wall nodes if the
% room connection classifier identifies the connection type to be ``door''
% (pre-condition). The rule generates a new wall node with a hole at the
% doorway, and adds these two new nodes with a door node in the
% middle. The location of the doorways are also given by the
% classifier. Since the wall node is a simple quad, the geometries of
% these new nodes can be obtained easily.


