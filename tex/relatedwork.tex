\section{Related Work}
This paper touches many topics in diverse fields from Computer
Vision, Computer Graphics and Robotics.
% general 3D reconstruction (Computer Vision), architectural scene
% modeling (Computer Vision and Computer Graphics), image segmentation and
% recognition (Computer Vision), and submodular optimization (Machine
% Learning).x
We here limit the discussion to closely related topics.

\mysubsubsection{Small-scale indoor modeling}
Indoor scene modeling has been active in Computer Vision, where a room
layout and/or object placements are inferred from an
image~\cite{Hedau2009,fouhey2013data}, an RGBD
image~\cite{Hoiem13,cornel_indoor_13}. Data driven approaches
with 3D model databases yield CAD-quality
reconstructions~\cite{cad_matching_12,search_classify_12}. However, the
focus of these methods is so far on a clutter analysis in a small scale
such as a single room.
% , their focus
% is mainly on analysis of the room structure in a small scale rather than
% complete geometry reconstruction of multiple rooms.
%
Semantic reconstruction (SR) from an RGBD stream has been active in
Robotics and Computer Vision~\cite{jia20133d,herbst2014toward}. These
methods handle slightly larger scenes, but are still not at the scale of
an entire scene with multiple rooms. Also, SR merely assigns semantics
to the pre-reconstructed 3D models. Segmentation, recognition, and
reconstruction are unified into a single formulation
in~\cite{hane2013joint}, but their operating range is also small, as the
entire scene is modeled as a voxel grid.

\mysubsubsection{Large-scale indoor modeling}
Recent developments in the sensor technology enable us to capture
dense 3D points in a large scale.
%In the domain of construction and civil engineering, automated
%acquisition of semantically rich building information models (BIMs) is a
%focus of interest. They analyze building-scale LiDAR models.
However, when it comes to the construction of a real 3D model (e.g., a
mesh), existing methods only produce a set of planar patches at a room
scale~\cite{xiong2013automatic}, simple primitives for a part of a
scene~\cite{mani_progress_monitoring}, a dense mesh from a voxel
grid~\cite{Turner2015}, an extruded model from a
floorplan~\cite{Turner2015}, or a ``polygon soup'' without any structure
or semantics~\cite{eccv_museum}.
%
%
For large scenes, room segmentation is also an important problem. The
current state-of-the-art methods solve in a top-down 2D
domain~\cite{Turner2015,Mura2014}, although room boundaries are often
ambiguous in a top-down view. We explicitly classify room connection
types with full 3D information.


\mysubsubsection{Architectural shape priors}
%Geometric regularities, such as planarity and orthogonality, have been
%used for man-made scenes.
Primitive detection is combined with Markov Random Field (MRF) to
reconstruct piecewise planar depthmaps~\cite{ManhattanWorldStereo}.
Detected 3D primitives can be assembled directly in the 3D
space~\cite{eccv_museum}.  However, excessive number of primitives need
to be extracted in general not to miss a single important one, and the
primitive detecion often becomes the most problematic step. Instead, we
combine MRF with Robust Principal Component Analysis
(RPCA)~\cite{Candes2011} to enforce both the low-entropy and the
low-rank structure in the depthmap to obtain much more compact 3D models
without extracting any primitive.

% which is exacerbated by the curse of dimensionality, where only very
% simple primitive types can be used.
%
%Primitive generation has been combined with a popular bottom-up
%formulation, Markov Random Field (MRF) for
%robustness~\cite{MRF_middlebury,MRF_evaluation_2013}.
%
%  For example, in a case of a depthmap estimation,
% one of the extracted primitives, instead of a discretized depth value,
% is assigned to each pixel

% These methods enforce geometric regularities fairly well, but the models
% are still a ``polygon soup'' without any structure or semantic
% information~\cite{ManhattanWorldStereo,GallupPlanar10,SinhaPlane09,lukas2008ffa}. They
% also lack in the ability to effectively control the model complexities.

%\mysubsubsection{Structured reconstruction} A seminal work by Dick et
%al.~\cite{Cipolla_MCMC} employs Markov Chain Monte Carlo (MCMC) to reconstruct
%structured outdoor building models. 2.5D facade
%models are reconstructed from images by following a shape
%grammar~\cite{ProceduralFacade}.
%% When it comes to building facades, which exhibit strong structural patterns,
%% 2.5D facade models are reconstructed from images by following shape
%% grammars
%The detection and parsing, although not necessarily the reconstruction,
%of outdoor buildings were demonstrated from LiDAR point clouds by
%applying a simple grammar~\cite{toshev2010detecting}. However, all these
%methods handle outdoor scenes, which are usually much simpler.
%and free from clutter.
%We have produced a recent breakthrough in indoor 3D modeling, where
%constructive solid geometry (CSG) is used to reconstruct large museum
%interior models~\cite{eccv_museum}.
% including Met museum in NYC, one of the
% largest art galleries in the world.

