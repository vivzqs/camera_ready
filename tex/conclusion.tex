\section{Limitations and Future Directions}

This paper presents a novel structured model representation and
reconstruction algorithm.
%, which is driven by the structure grammar defining the structural
%relationships of scene elements.
We admit that our current system has limitations. First, the most
problematic step is the connection classification. Our current approach
relies only on local geometric information, and tends to under-segment.
% Our classification rule is fairly complex and a future work is to
A future work is to exploit images to robustify the classification
process. Scene and object recognition accuracies are also poor at many
places, where building-scale context from the structure graph may
help. For example, a bedroom is unlikely to be in the middle of offices.
%structured scene representation at the scale of an entire building, much
%bigger than a single image and a panorama.
%structural regularities and patterns are persistent abundant in the same building.
%Our structured model representation provides much bigger context in a
%building scale, which has a potential to further boost the
%performance.
%%For example, in the same building,
%chairs and tables tend to have the same shape, and
%%windows tend to have the same shape and be at the same height.
%(for object detection).
% It is unlikely to see a bedroom in the middle of conference
%rooms (for scene recognition).
Another future work is to test our framework on RGBD stream datasets as
opposed to panorama RGBD images. Although the camera pose quality tends
to degrade for RGBD streams, our framework should be applicable.
% We have chosen to work on the latter for several reasons. First, the
% camera pose quality tends to degrade for RGBD streams, especially for
% challenging scenes.  Second, the panorama experience is becoming an
% industry standard for scene visualization~\cite{google-maps,matterport},
% which is not easy to achieve from RGBD streams.
% %the lack of an ability to turn around exposes significant drawbacks in
% %real applications.

We hope that
%The datasets and the source code will be shared with the research
%community.
this paper opens up a number of new exciting research directions such as
the following, and share the soruce code and the datasets at the project
website~\cite{SIMProject}.
% and facilitates novel industrial applications.

\mysubsubsubsection{Structured X modeling}
Structured modeling is a general framework, and should be applicable to
outdoor scenes or man-made objects.
%Structured modeling framework should be applicable outside the
%indoor scenes. Structured outdoor modeling or structured object modeling
%is an interesting space to also explore.
%

\mysubsubsubsection{Indoor Inverse Procedural Modeling}
Structure grammar is manually constructed in our experiments, where an
interesting alternative is to learn the grammar from examples.

\mysubsubsubsection{Inverse-CAD and Scan2BIM} Inverse-CAD has been a
long-standing research milestone in Computer Vision, Computer Graphics,
and Civil Engineering.  For major architectural components, our models
already enable practical human post-processing. This research could
serve as a good stepping stone to tackle an even harder Scan2BIM
problem.

\yasu{a bit strange finish}
